1. Clase HashMap
Esta clase crea y maneja un mapa hash. Tiene métodos para agregar, obtener, eliminar, verificar y limpiar datos.

2. Constructor
javascript
Copy code
constructor(size = 16) {
  this.buckets = [];
  for (let i = 0; i < size; i++) {
    this.buckets.push([]); // Añade un array vacío para cada "cubo"
  }
  this.size = 0; // Número de elementos en el mapa
}
Propósito: Configura los cubos (buckets) donde se almacenarán los datos.
Funcionamiento:
Crea un array this.buckets de tamaño size (por defecto, 16).
Cada elemento de this.buckets es un array vacío que servirá como un cubo para almacenar pares clave-valor.
this.size lleva la cuenta de cuántos pares clave-valor hay en el mapa.
3. Método hash(key)
javascript
Copy code
hash(key) {
  let hashCode = 0;
  for (let i = 0; i < key.length; i++) {
    hashCode += key.charCodeAt(i); // Suma los valores ASCII de los caracteres
  }
  return hashCode % this.buckets.length; // Ajusta al tamaño de los cubos
}
Propósito: Convierte una clave (string) en un índice numérico que indica en qué cubo almacenar el dato.
Funcionamiento:
Calcula un número (hashCode) sumando los códigos ASCII de los caracteres de la clave.
Ajusta el resultado (hashCode % this.buckets.length) para asegurarse de que esté dentro del rango de índices válidos de los cubos.
4. Método set(key, value)
javascript
Copy code
set(key, value) {
  const index = this.hash(key);
  const bucket = this.buckets[index];

  for (let entry of bucket) {
    if (entry.key === key) {
      entry.value = value; // Actualiza el valor
      return;
    }
  }

  bucket.push({ key, value });
  this.size++;
}
Propósito: Agrega o actualiza un par clave-valor en el mapa.
Funcionamiento:
Calcula el índice del cubo con hash(key).
Busca en el cubo si la clave ya existe:
Sí: Actualiza el valor.
No: Agrega un nuevo objeto { key, value } al cubo y aumenta this.size.
5. Método get(key)
javascript
Copy code
get(key) {
  const index = this.hash(key);
  const bucket = this.buckets[index];

  for (let entry of bucket) {
    if (entry.key === key) {
      return entry.value; // Devuelve el valor
    }
  }

  return null; // Si no lo encuentra, devuelve null
}
Propósito: Busca un valor dado su clave.
Funcionamiento:
Calcula el índice del cubo con hash(key).
Recorre el cubo correspondiente buscando un objeto con la clave deseada.
Devuelve el valor si lo encuentra; si no, devuelve null.
6. Método remove(key)
javascript
Copy code
remove(key) {
  const index = this.hash(key);
  const bucket = this.buckets[index];

  for (let i = 0; i < bucket.length; i++) {
    if (bucket[i].key === key) {
      bucket.splice(i, 1); // Elimina el elemento
      this.size--;
      return true; // Confirma la eliminación
    }
  }

  return false; // No se encontró la clave
}
Propósito: Elimina un par clave-valor del mapa.
Funcionamiento:
Calcula el índice del cubo con hash(key).
Busca la clave en el cubo.
Si la encuentra:
La elimina con splice.
Reduce this.size.
Devuelve true.
Si no la encuentra, devuelve false.
7. Método has(key)
javascript
Copy code
has(key) {
  return this.get(key) !== null;
}
Propósito: Verifica si una clave existe en el mapa.
Funcionamiento: Usa get para buscar la clave y devuelve true si encuentra un valor; de lo contrario, devuelve false.
8. Método clear()
javascript
Copy code
clear() {
  this.buckets = Array(this.buckets.length).fill(null).map(() => []);
  this.size = 0;
}
Propósito: Limpia todos los datos del mapa.
Funcionamiento:
Crea un nuevo array de cubos vacíos con el mismo tamaño que el actual.
Reinicia this.size a 0.
9. Métodos para obtener claves, valores y pares clave-valor
keys()
javascript
Copy code
keys() {
  const keys = [];
  for (let bucket of this.buckets) {
    for (let entry of bucket) {
      keys.push(entry.key);
    }
  }
  return keys;
}
Devuelve: Todas las claves como un array.
values()
javascript
Copy code
values() {
  const values = [];
  for (let bucket of this.buckets) {
    for (let entry of bucket) {
      values.push(entry.value);
    }
  }
  return values;
}
Devuelve: Todos los valores como un array.
entries()
javascript
Copy code
entries() {
  const entries = [];
  for (let bucket of this.buckets) {
    for (let entry of bucket) {
      entries.push([entry.key, entry.value]);
    }
  }
  return entries;
}
Devuelve: Todos los pares clave-valor como un array de arrays.
10. Pruebas
Agregar datos:

javascript
Copy code
test.set('apple', 'red');
Guarda "apple": "red" en el mapa.

Sobrescribir datos:

javascript
Copy code
test.set('apple', 'green');
Actualiza el valor de "apple" a "green".

Buscar datos:

javascript
Copy code
console.log(test.get('apple')); // 'green'
Eliminar datos:

javascript
Copy code
test.remove('banana');
Obtener todas las claves, valores o pares:

javascript
Copy code
console.log(test.keys()); // ['apple', 'grape']